\documentclass[landscape]{baposter}

\usepackage[vlined]{algorithm2e}
\usepackage{times}
\usepackage{calc}
\usepackage{url}
\usepackage{graphicx}
\usepackage{amsmath}
\usepackage{amssymb}
\usepackage{amsthm}
\usepackage{relsize}
\usepackage{multirow}
\usepackage{booktabs}

\usepackage{multicol}
\usepackage[T1]{fontenc}
\usepackage{ae}

\usepackage{colortbl}
%\usepackage[upright]{fourier}
%\usepackage{tkz-berge}
%\usepackage{tkz-graph}

% Define highlight box
\newcommand{\highlightbox}[1]{%
\begin{center}
\fcolorbox{red}{cyan!10!white}{%
\begin{minipage}{0.95\textwidth}
\centering
\bf #1%
\end{minipage}}
\end{center}}

\graphicspath{{images/}}

 %%%%%%%%%%%%%%%%%%%%%%%%%%%%%%%%%%%%%%%%%%%%%%%%%%%%%%%%%%%%%%%%%%%%%%%%%%%%%%%%
 % Format 
 \newcommand{\RotUP}[1]{\begin{sideways}#1\end{sideways}}
 \newcommand{\cyl}{\cellcolor[gray]{0.8}}
 %%%%%%%%%%%%%%%%%%%%%%%%%%%%%%%%%%%%%%%%%%%%%%%%%%%%%%%%%%%%%%%%%%%%%%%%%%%%%%%%
 % Multicol Settings
 %%%%%%%%%%%%%%%%%%%%%%%%%%%%%%%%%%%%%%%%%%%%%%%%%%%%%%%%%%%%%%%%%%%%%%%%%%%%%%%%
 \setlength{\columnsep}{0.7em}
 \setlength{\columnseprule}{0mm}
 %%%%%%%%%%%%%%%%%%%%%%%%%%%%%%%%%%%%%%%%%%%%%%%%%%%%%%%%%%%%%%%%%%%%%%%%%%%%%%%%
 % Save space in lists. Use this after the opening of the list
 %%%%%%%%%%%%%%%%%%%%%%%%%%%%%%%%%%%%%%%%%%%%%%%%%%%%%%%%%%%%%%%%%%%%%%%%%%%%%%%%
 \newcommand{\compresslist}{%
 \setlength{\itemsep}{1pt}%
 \setlength{\parskip}{0pt}%
 \setlength{\parsep}{0pt}%
 }
 %%%%%%%%%%%%%%%%%%%%%%%%%%%%%%%%%%%%%%%%%%%%%%%%%%%%%%%%%%%%%%%%%%%%%%%%%%%%%%
 % Formating
 \newcommand{\Matrix}[1]{\begin{bmatrix} #1 \end{bmatrix}}
 \newcommand{\Vector}[1]{\begin{pmatrix} #1 \end{pmatrix}}

 \newcommand*{\norm}[1]{\mathopen\| #1 \mathclose\|}% use instead of $\|x\|$
 \newcommand*{\abs}[1]{\mathopen| #1 \mathclose|}% use instead of $\|x\|$
 \newcommand*{\normLR}[1]{\left\| #1 \right\|}% use instead of $\|x\|$

 \newcommand*{\SET}[1]  {\ensuremath{\mathcal{#1}}}
 \newcommand*{\FUN}[1]  {\ensuremath{\mathcal{#1}}}
 \newcommand*{\MAT}[1]  {\ensuremath{\boldsymbol{#1}}}
 \newcommand*{\VEC}[1]  {\ensuremath{\boldsymbol{#1}}}
 \newcommand*{\CONST}[1]{\ensuremath{\mathit{#1}}}

 \DeclareMathOperator*{\argmax}{arg\,max}
 \DeclareMathOperator*{\diag}{diag}
 \DeclareMathOperator*{\argmin}{arg\,min}
 \DeclareMathOperator*{\vectorize}{vec}
 \DeclareMathOperator*{\reshape}{reshape}
  \DeclareMathOperator{\E}{E}
   \DeclareMathOperator{\Corr}{Corr}
   \DeclareMathOperator{\J}{J}

\newtheorem{definition}{Definition}
\newtheorem{thm}{Theorem}

%%%%%%%%%%%%%%%%%%%%%%%%%%%%%%%%%%%%%%%%%%%%%%%%%%%%%%%%%%%%%%%%%%%%%%%%%%%%%
%% Begin of Document
%%%%%%%%%%%%%%%%%%%%%%%%%%%%%%%%%%%%%%%%%%%%%%%%%%%%%%%%%%%%%%%%%%%%%%%%%%%%%
\begin{document}
%%%%%%%%%%%%%%%%%%%%%%%%%%%%%%%%%%%%%%%%%%%%%%%%%%%%%%%%%%%%%%%%%%%%%%%%%%%%%
%% Here starts the poster
%%---------------------------------------------------------------------------
%% Format it to your taste with the options
%%%%%%%%%%%%%%%%%%%%%%%%%%%%%%%%%%%%%%%%%%%%%%%%%%%%%%%%%%%%%%%%%%%%%%%%%%%%%
\begin{poster}{
 % Show grid to help with alignment
 grid=no,
 % Column spacing
 colspacing=0.7em,
 % Color style
 headerColorOne=cyan!20!white!90!black,
 borderColor=cyan!30!white!90!black,
 % Format of textbox
 textborder=faded,
 % Format of text header
 eyecatcher=no,
 headerborder=open,
 headershape=roundedright,
 background=none,
 headerfont=\Large\textsf, %Sans Serif
 bgColorOne=cyan!10!white,
 linewidth=2pt,
 headerheight=0.1\textheight}
 % Eye Catcher
 {

 }
 % Title
 {\sf Suboptimal Provision of Privacy and Statistical Accuracy \newline When They are Public Goods}
 % Authors
 {
 %\sf John M.\ Abowd, Ian M.\ Schmutte, William Sexton, Lars Vilhuber
 \sf J. M.\ Abowd, I. M.\ Schmutte, W. Sexton, L. Vilhuber
 %\newline 
 \hspace{3em}
 {Correspondence: \texttt{william.n.sexton@census.gov}}
 }
 % University logo
 {
  \begin{tabular}{rrr}
    \includegraphics[height=0.04\textheight]{census} &
    \includegraphics[height=0.04\textheight]{thick4c} &
    \includegraphics[height=0.04\textheight]{CULogo4c}
  \end{tabular}
 }

%%%%%%%%%%%%%%%%%%%%%%%%%%%%%%%%%%%%%%%%%%%%%%%%%%%%%%%%%%%%%%%%%%%%%%%%%%%%%%
%%% Now define the boxes that make up the poster
%%%---------------------------------------------------------------------------
%%% Each box has a name and can be placed absolutely or relatively.
%%% The only inconvenience is that you can only specify a relative position 
%%% towards an already declared box. So if you have a box attached to the 
%%% bottom, one to the top and a third one which should be inbetween, you 
%%% have to specify the top and bottom boxes before you specify the middle 
%%% box.
%%%%%%%%%%%%%%%%%%%%%%%%%%%%%%%%%%%%%%%%%%%%%%%%%%%%%%%%%%%%%%%%%%%%%%%%%%%%%%

%%%%%%%%%%%%%%%%%%%%%%%%%%%%%%%%%%%%%%%%%%%%%%%%%%%%%%%%%%%%%%%%%%%%%%%%%%%%%%
  \headerbox{I.\ Goal and Contribution}{name=contribution,column=0,row=0,span=1.1}{
%%%%%%%%%%%%%%%%%%%%%%%%%%%%%%%%%%%%%%%%%%%%%%%%%%%%%%%%%%%%%%%%%%%%%%%%%%%%%%
\highlightbox{Goal: To explain why population statistics are provided by public statistical agencies rather than private firms}

To do so, we focus on inefficiencies in how private providers trade off data privacy and accuracy.

	\begin{itemize}
		\item Increasing the accuracy of published statistical summaries necessarily results in a loss of privacy for the data owners.
		\item Data publication is based on differential privacy.
		\item Privacy protection and accuracy are public goods.
	\end{itemize}
% Start highlight box
\highlightbox{%
We find that private provision results in suboptimally low data accuracy.
}

% end highlight box
		\begin{itemize}
			\item The external benefit of data accuracy to all consumers is not captured by the willingness-to-pay of the consumer with the greatest private value.
			\item The provider buys just enough data-use rights (privacy loss) to sell the data accuracy to the consumer with the highest valuation.
		\end{itemize}
	
  
  }
%%%%%%%%%%%%%%%%%%%%%%%%%%%%%%%%%%%%%%%%%%%%%%%%%%%%%%%%%%%%%%%%%%%%%%%%%%%%%%
  \headerbox{II.\ Modeling Privacy and Accuracy}{name=endmob,column=0,span=1.1,below=contribution}{
%%%%%%%%%%%%%%%%%%%%%%%%%%%%%%%%%%%%%%%%%%%%%%%%%%%%%%%%%%%%%%%%%%%%%%%%%%%%%%
%\textbf{Data privacy} and \textbf{data accuracy} are quantified as follows: %Differentially private data publications do not allow an outsider to learn ``too much'' about any individual data record based on statistical summaries of the full database. 

\underline{\large \bf $\varepsilon$-differential privacy:}
\begin{definition} Query release mechanism $M$ satisfies $\varepsilon$%
-differential privacy if for $\varepsilon > 0$, for all pairs of neighboring databases $D,D^{\prime }$, all queries $Q\in\mathcal{Q}$, and all $B\in \mathcal{B}$
\begin{equation*}
\Pr \left[ M(D,Q)\in B |D,Q\right] \leq e^{\varepsilon }\Pr \left[
M(D^{\prime },Q)\in B |D^{\prime},Q\right],
\end{equation*}%
where $\mathcal{B}$ are the measurable subsets of $\mathbb{R}$, 
and the randomness in $M$ is due exclusively to the mechanism.
%and not the process generating the database $D$.
\end{definition}
%\textbf{Data accuracy} is quantified as follows:

\underline{\large \bf $(\alpha ,\protect\beta )$-accuracy:}
\begin{definition}
 Query release mechanism $M$ satisfies $(\alpha ,\beta )$%
-accuracy if for $Q\in\mathcal{Q}$ and $a$ output from $M(D,Q)$,
$$\mbox{Pr}\Big( \, |a-Q(D)| \le \alpha \; \Big| \; D,Q\Big) \; \ge \; 1-\beta $$
where $a, Q(D) \in \mathbb{R}$.
\end{definition}


}
%%%%%%%%%%%%%%%%%%%%%%%%%%%%%%%%%%%%%%%%%%%%%%%%%%%%%%%%%%%%%%%%%%%%%%%%%%%%%%
%  \headerbox{Our Approach}{name=disclaimer,column=0,above=bottom}{
%%%%%%%%%%%%%%%%%%%%%%%%%%%%%%%%%%%%%%%%%%%%%%%%%%%%%%%%%%%%%%%%%%%%%%%%%%%%%%
%
%}

 %%%%%%%%%%%%%%%%%%%%%%%%%%%%%%%%%%%%%%%%%%%%%%%%%%%%%%%%%%%%%%%%%%%%%%%%%%%%%%

   \headerbox{V.Competitive Market Equilibrium}{name=method,column=2.1,span=1.9,row=0, }{
 %%%%%%%%%%%%%%%%%%%%%%%%%%%%%%%%%%%%%%%%%%%%%%%%%%%%%%%%%%%%%%%%%%%%%%%%%%%%%%
 
A private profit-maximizing, price-taking, firm sells $\hat{s}$ with data accuracy $I$ at price $p_{I}$. Then, profits $P\left( I\right) $ are%
\begin{equation*}
    P\left( I\right) =p_{I}I-C^{VCG}(I).
\end{equation*}%
If it sells at all, it will produce $I$ to satisfy the first-order condition
$P^{\prime }\left( I^{VCG}\right) =0$ implying%
\begin{equation}
    p_{I}=Q\left( \frac{H(I)}{N}\right) H(I)\varepsilon ^{\prime }(I)+\left[
    Q\left( \frac{H(I)}{N}\right) +Q^{\prime }\left( \frac{H(I)}{N}\right)
    \left( \frac{H(I)}{N}\right) \right] H^{\prime }(I)\varepsilon (I)
    \label{eq:p_private}
\end{equation}%
where the solution is evaluated at $I^{VCG}$. %The price of data accuracy is
%equal to the marginal cost of increasing the amount of privacy
%protection--data-use rights--that must be purchased. There are two terms.
%The first term is the increment to marginal cost from increasing the amount
%each privacy-right seller must be paid because $\varepsilon $ has been
%marginally increased, thus reducing privacy protection for all. The second
%term is the increment to marginal cost from increasing the number of people
%from whom data-use rights with privacy protection $\varepsilon $ must be
%purchased.
As long as the cost function is strictly increasing and convex,
the existence and uniqueness of a solution is guaranteed. 

At market price $p_{I}$, consumer $i$'s willingness to pay for data accuracy will be given by solving%
\begin{equation*}
    \max_{I_{i}\geq 0}\eta _{i}\left( I^{\symbol{126}i}+I_{i}\right) -p_{I}I_{i}.
\end{equation*} 

%If there exists at least one consumer for whom $\eta _{i}\geq p_{I}$, then the solution to
%equation (\ref{eq:p_private}) is attained for $I^{VCG}>0.$

Consumers are playing a classic free-rider game.  
\begin{enumerate}
    \item the only person willing to pay for the public good is one with the maximum value of $%
\eta _{i}$.
    \item all others will purchase zero data accuracy but still consume the data accuracy purchased by this lone consumer. 
\end{enumerate}
Hence, equilibrium price and data accuracy will satisfy%
\begin{equation*}
    p_{I}=\bar{\eta}=\frac{dC^{VCG}\left( I^{VCG}\right) }{dI}, \text{ where } \bar{\eta} = \max \eta_i. 
\end{equation*}%
%where $\bar{\eta}$ is the maximum value of $\eta _{i}$ in the
%population--the taste for accuracy of the person who desires it the most.
However, the Pareto optimal consumption of data accuracy, $I^{0},$ solves
\begin{equation}
    \sum_{i=1}^{N}\eta _{i}=\frac{dC^{VCG}\left( I^{0}\right) }{dI}.
    \label{eqn:pareto_optimality}
\end{equation}%
Marginal cost is positive, $\frac{dC^{VCG}\left( I^{0}\right) }{dI}>0$, and $%
\sum_{i=1}^{N}\eta _{i} > \bar{\eta}$; therefore, data accuracy will be
under-provided by a competitive supplier when data accuracy is a public good
as long as marginal cost is increasing. More
succinctly, $I^{VCG} < I^{0}$. Therefore, privacy protection must be
over-provided, $\varepsilon ^{VCG} < \varepsilon ^{0}$.





 
   }
   
 %%%%%%%%%%%%%%%%%%%%%%%%%%%%%%%%%%%%%%%%%%%%%%%%%%%%%%%%%%%%%%%%%%%%%%%%%%%%%
   % \headerbox{Future Work}{name=else,column=2,span=1,above=bottom}{
 %%%%%%%%%%%%%%%%%%%%%%%%%%%%%%%%%%%%%%%%%%%%%%%%%%%%%%%%%%%%%%%%%%%%%%%%%%%%%
   % Something else.}
   


 %%%%%%%%%%%%%%%%%%%%%%%%%%%%%%%%%%%%%%%%%%%%%%%%%%%%%%%%%%%%%%%%%%%%%%%%%%%%%%
   \headerbox{VI.\ Suboptimality Theorem}{name=correction,column=2.1,below=method,above=bottom,span=1.9}{
 %%%%%%%%%%%%%%%%%%%%%%%%%%%%%%%%%%%%%%%%%%%%%%%%%%%%%%%%%%%%%%%%%%%%%%%%%%%%%
\begin{thm}
\label{theorem:suboptimality}If preferences are as in III, the query response mechanism and cost function for the VCG mechanism are as displayed in IV, 
the population distribution of $\gamma $ is given by $F_{\gamma }$ (bounded,
absolutely continuous, everywhere differentiable, and with quantile function
$Q$ satisfying the conditions such that \ref{eq:p_private} has a solution,
the population distribution of $\eta$ has bounded support on $\left[ 0,\bar{%
\eta}\right] $, and the population in the database is represented as a
continuum with measure function $H$ (absolutely continuous, everywhere
differentiable, and with total measure $N$) then $I^{VCG} < I^{0}$, where $%
I^{0}$ is the Pareto optimal level of $I$ solving equation $\left( \ref%
{eqn:pareto_optimality}\right) $,
and $I^{VCG}$ is the privately-provided level when using the VCG\
procurement mechanism.
\end{thm}
}

 %%%%%%%%%%%%%%%%%%%%%%%%%%%%%%%%%%%%%%%%%%%%%%%%%%%%%%%%%%%%%%%%%%%%%%%%%%%%%%
   \headerbox{III.\ Model (Consumer)}{name=rmn,column=1.1,row=0}{
 %%%%%%%%%%%%%%%%%%%%%%%%%%%%%%%%%%%%%%%%%%%%%%%%%%%%%%%%%%%%%%%%%%%%%%%%%%%%%%
 There are $N$ private individuals:
 \begin{itemize}
 \item each possesses a single bit of information, $b_{i}$, and is endowed with
income, $y_{i}$.
     \item each consume one unit of the published statistic, which has
accuracy $I=(1-\alpha )$. Each is charged at the market price $p_{I}$, for her \textquotedblleft
share\textquotedblright\ of $I$, denoted $I_{i}$.

     \item preferences are given by the indirect utility function
\begin{align*}
v_{i}\left( y_{i},\varepsilon _{i},I_{i},I^{\symbol{126}i}\right) =&\ln
y_{i}+p_{\varepsilon }\varepsilon _{i}-\gamma _{i}\varepsilon _{i} \\ &+\eta
_{i}\left( I_{i}+I^{\symbol{126}i}\right) -p_{I}I_{i}.  \label{eqn:linear2}
\end{align*}%
%Equation~(\ref{eqn:linear2}) implies that preferences are quasilinear in
%data accuracy, $I$, privacy loss, $\varepsilon _{i}$, and log income, $\ln
%y_{i}$.
\end{itemize}
The term $%
p_{\varepsilon }$ is the common price per unit of privacy. And, $\left( \eta_{i},\gamma _{i}\right) >0,$ are the individual's marginal
preferences for data accuracy and privacy loss and are not known to the
data provider, but their population distributions are public information.
 %Prices are determined by the model.
   }
%The evolution of the realized mobility network is a function of observable and latent node and edge characteristics.


 %%%%%%%%%%%%%%%%%%%%%%%%%%%%%%%%%%%%%%%%%%%%%%%%%%%%%%%%%%%%%%%%%%%%%%%%%%%
   % \headerbox{Future Work}{name=future,column=2,span=2,above=bottom}{
 %%%%%%%%%%%%%%%%%%%%%%%%%%%%%%%%%%%%%%%%%%%%%%%%%%%%%%%%%%%%%%%%%%%%%%%%%%%

   
   % These methods must be scaled up for application to a large matched employer-employee database.
% \begin{itemize}
	% \item   600 billion wage records. 
	% \item 200 million workers. 
	% \item 10 million employers. 
% \end{itemize}
  
   % }

 %%%%%%%%%%%%%%%%%%%%%%%%%%%%%%%%%%%%%%%%%%%%%%%%%%%%%%%%%%%%%%%%%%%%%%%%%%%%%%%
  \headerbox{IV.\ Model (Producer)}{name=model,column=1.1,span=1,below=rmn}{
 %%%%%%%%%%%%%%%%%%%%%%%%%%%%%%%%%%%%%%%%%%%%%%%%%%%%%%%%%%%%%%%%%%%%%%%%%%%%%%%
 
Ghosh and Roth (2015) prove that publishing
\begin{equation*}
\hat{s}=\frac{1}{N}\left[ \sum_{i=1}^{H}b_{i}+\frac{\alpha N}{2\left(1/2+\ln \frac{1}{\beta} \right)} +%
Lap\left( \frac{1}{\varepsilon }\right) \right] \label{eqn:GRquery}
\end{equation*}%
gives an $\left( \alpha ,\beta\right) $-accuracy estimate of the population mean, $\bar{b}$,
requiring privacy loss $%
\varepsilon_{i}=\varepsilon (I)=\frac{1/2+\ln{\left(1/\beta\right)}}{(1-I)N}$ from $H(I)=N-\frac{(1-I)N}{1/2+\ln{\left(1/\beta\right)}}$ members of the population.
\begin{itemize}
\item Purchasing data-use rights from those with the
smallest $\gamma _{i}$, is a minimum-cost, envy-free VCG mechanism.
\end{itemize}
Under said VCG mechanism, the total cost of producing $I$ is
\begin{equation*}
    C^{VCG}(I)=p_\varepsilon H(I)\varepsilon (I)=Q\left( \frac{H(I)}{N}\right) H(I)\varepsilon (I)
\end{equation*} where $Q$ is the quantile function with
respect to the population distribution of privacy preferences, $F_{\gamma }$.
}
\end{poster}%
%
\end{document}
